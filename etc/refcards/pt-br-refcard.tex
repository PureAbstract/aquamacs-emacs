% Reference Card for GNU Emacs version 23 on Unix systems
%**start of header
\newcount\columnsperpage
\newcount\letterpaper

% This file can be printed with 1, 2, or 3 columns per page (see below).
% Specify how many you want here.

\columnsperpage=3

% Set letterpaper to 0 for A4 paper, 1 for letter (US) paper.  Useful
% only when columnsperpage is 2 or 3.

\letterpaper=0

% PDF output layout.  0 for A4, 1 for letter (US), a `l' is added for
% a landscape layout.

\input pdflayout.sty
\pdflayout=(0l)

% Nothing else needs to be changed below this line.
% Copyright (C) 1987, 1993, 1996, 1997, 2002, 2003, 2004,
%   2006, 2007, 2008, 2009  Free Software Foundation, Inc.

% This file is part of GNU Emacs.

% GNU Emacs is free software: you can redistribute it and/or modify
% it under the terms of the GNU General Public License as published by
% the Free Software Foundation, either version 3 of the License, or
% (at your option) any later version.

% GNU Emacs is distributed in the hope that it will be useful,
% but WITHOUT ANY WARRANTY; without even the implied warranty of
% MERCHANTABILITY or FITNESS FOR A PARTICULAR PURPOSE.  See the
% GNU General Public License for more details.

% You should have received a copy of the GNU General Public License
% along with GNU Emacs.  If not, see <http://www.gnu.org/licenses/>.

% This file is intended to be processed by plain TeX (TeX82).
%
% The final reference card has six columns, three on each side.
% This file can be used to produce it in any of three ways:
% 1 column per page
%    produces six separate pages, each of which needs to be reduced to 80%.
%    This gives the best resolution.
% 2 columns per page
%    produces three already-reduced pages.
%    You will still need to cut and paste.
% 3 columns per page
%    produces two pages which must be printed sideways to make a
%    ready-to-use 8.5 x 11 inch reference card.
%    For this you need a dvi device driver that can print sideways.
% Which mode to use is controlled by setting \columnsperpage above.
%
% To compile and print this document:
% tex refcard.tex
% dvips -t landscape refcard.dvi
%
% Author:
%  Stephen Gildea
%  Internet: gildea@stop.mail-abuse.org
%
% Thanks to Paul Rubin, Bob Chassell, Len Tower, and Richard Mlynarik
% for their many good ideas.

% If there were room, it would be nice to see a section on Dired.

% Translated from English to Portuguese by Rodrigo Real, send comments
% and suggestions to rreal@ucpel.tche.br. Thanks to Mario Goulart for
% the opinions.


\def\versionnumber{2.3}
\def\versionyear{2006}          % latest update
\def\versionemacs{23}
\def\year{2009}                 % latest copyright year

\def\shortcopyrightnotice{\vskip 1ex plus 2 fill
  \centerline{\small \copyright\ \year\ Free Software Foundation, Inc.
  Permissions on back.  v\versionnumber}}

\def\copyrightnotice{
\vskip 1ex plus 2 fill\begingroup\small
\centerline{Copyright \copyright\ \year\ Free Software Foundation, Inc.}
\centerline{v\versionnumber{} for GNU Emacs version \versionemacs, \versionyear}
\centerline{designed by Stephen Gildea}

Permission is granted to make and distribute copies of
this card provided the copyright notice and this permission notice
are preserved on all copies.

For copies of the GNU Emacs manual, write to the Free Software
Foundation, Inc., 51 Franklin Street, Fifth Floor, Boston, MA  02110-1301  USA

\endgroup}

% make \bye not \outer so that the \def\bye in the \else clause below
% can be scanned without complaint.
\def\bye{\par\vfill\supereject\end}

\newdimen\intercolumnskip	%horizontal space between columns
\newbox\columna			%boxes to hold columns already built
\newbox\columnb

\def\ncolumns{\the\columnsperpage}

\message{[\ncolumns\space
  column\if 1\ncolumns\else s\fi\space per page]}

\def\scaledmag#1{ scaled \magstep #1}

% This multi-way format was designed by Stephen Gildea October 1986.
% Note that the 1-column format is fontfamily-independent.
\if 1\ncolumns			%one-column format uses normal size
  \hsize 4in
  \vsize 10in
  \voffset -.7in
  \font\titlefont=\fontname\tenbf \scaledmag3
  \font\headingfont=\fontname\tenbf \scaledmag2
  \font\smallfont=\fontname\sevenrm
  \font\smallsy=\fontname\sevensy

  \footline{\hss\folio}
  \def\makefootline{\baselineskip10pt\hsize6.5in\line{\the\footline}}
\else				%2 or 3 columns uses prereduced size
  \hsize 3.2in
  \if 1\the\letterpaper
     \vsize 7.95in
  \else
     \vsize 7.65in
  \fi
  \hoffset -.75in
  \voffset -.745in
  \font\titlefont=cmbx10 \scaledmag2
  \font\headingfont=cmbx10 \scaledmag1
  \font\smallfont=cmr6
  \font\smallsy=cmsy6
  \font\eightrm=cmr8
  \font\eightbf=cmbx8
  \font\eightit=cmti8
  \font\eighttt=cmtt8
  \font\eightmi=cmmi8
  \font\eightsy=cmsy8
  \textfont0=\eightrm
  \textfont1=\eightmi
  \textfont2=\eightsy
  \def\rm{\eightrm}
  \def\bf{\eightbf}
  \def\it{\eightit}
  \def\tt{\eighttt}
  \if 1\the\letterpaper
     \normalbaselineskip=.8\normalbaselineskip
  \else
     \normalbaselineskip=.7\normalbaselineskip
  \fi
  \normallineskip=.8\normallineskip
  \normallineskiplimit=.8\normallineskiplimit
  \normalbaselines\rm		%make definitions take effect

  \if 2\ncolumns
    \let\maxcolumn=b
    \footline{\hss\rm\folio\hss}
    \def\makefootline{\vskip 2in \hsize=6.86in\line{\the\footline}}
  \else \if 3\ncolumns
    \let\maxcolumn=c
    \nopagenumbers
  \else
    \errhelp{You must set \columnsperpage equal to 1, 2, or 3.}
    \errmessage{Illegal number of columns per page}
  \fi\fi

%%  \intercolumnskip=.46in
  \intercolumnskip=.65in
  \def\abc{a}
  \output={%			%see The TeXbook page 257
      % This next line is useful when designing the layout.
      %\immediate\write16{Column \folio\abc\space starts with \firstmark}
      \if \maxcolumn\abc \multicolumnformat \global\def\abc{a}
      \else\if a\abc
	\global\setbox\columna\columnbox \global\def\abc{b}
        %% in case we never use \columnb (two-column mode)
        \global\setbox\columnb\hbox to -\intercolumnskip{}
      \else
	\global\setbox\columnb\columnbox \global\def\abc{c}\fi\fi}
  \def\multicolumnformat{\shipout\vbox{\makeheadline
      \hbox{\box\columna\hskip\intercolumnskip
        \box\columnb\hskip\intercolumnskip\columnbox}
      \makefootline}\advancepageno}
  \def\columnbox{\leftline{\pagebody}}

  \def\bye{\par\vfill\supereject
    \if a\abc \else\null\vfill\eject\fi
    \if a\abc \else\null\vfill\eject\fi
    \end}
\fi

% we won't be using math mode much, so redefine some of the characters
% we might want to talk about
\catcode`\^=12
\catcode`\_=12

\chardef\\=`\\
\chardef\{=`\{
\chardef\}=`\}

\hyphenation{mini-buf-fer}

\parindent 0pt
\parskip 1ex plus .5ex minus .5ex

\def\small{\smallfont\textfont2=\smallsy\baselineskip=.8\baselineskip}

% newcolumn - force a new column.  Use sparingly, probably only for
% the first column of a page, which should have a title anyway.
\outer\def\newcolumn{\vfill\eject}

% title - page title.  Argument is title text.
\outer\def\title#1{{\titlefont\centerline{#1}}\vskip 1ex plus .5ex}

% section - new major section.  Argument is section name.
\outer\def\section#1{\par\filbreak
  \vskip 3ex plus 2ex minus 2ex {\headingfont #1}\mark{#1}%
  \vskip 2ex plus 1ex minus 1.5ex}

\newdimen\keyindent

% beginindentedkeys...endindentedkeys - key definitions will be
% indented, but running text, typically used as headings to group
% definitions, will not.
\def\beginindentedkeys{\keyindent=1em}
\def\endindentedkeys{\keyindent=0em}
\endindentedkeys

% paralign - begin paragraph containing an alignment.
% If an \halign is entered while in vertical mode, a parskip is never
% inserted.  Using \paralign instead of \halign solves this problem.
\def\paralign{\vskip\parskip\halign}

% \<...> - surrounds a variable name in a code example
\def\<#1>{{\it #1\/}}

% kbd - argument is characters typed literally.  Like the Texinfo command.
\def\kbd#1{{\tt#1}\null}	%\null so not an abbrev even if period follows

% beginexample...endexample - surrounds literal text, such a code example.
% typeset in a typewriter font with line breaks preserved
\def\beginexample{\par\leavevmode\begingroup
  \obeylines\obeyspaces\parskip0pt\tt}
{\obeyspaces\global\let =\ }
\def\endexample{\endgroup}

% key - definition of a key.
% \key{description of key}{key-name}
% prints the description left-justified, and the key-name in a \kbd
% form near the right margin.
\def\key#1#2{\leavevmode\hbox to \hsize{\vtop
  {\hsize=.75\hsize\rightskip=1em
  \hskip\keyindent\relax#1}\kbd{#2}\hfil}}

\newbox\metaxbox
\setbox\metaxbox\hbox{\kbd{M-x }}
\newdimen\metaxwidth
\metaxwidth=\wd\metaxbox

% metax - definition of a M-x command.
% \metax{description of command}{M-x command-name}
% Tries to justify the beginning of the command name at the same place
% as \key starts the key name.  (The "M-x " sticks out to the left.)
\def\metax#1#2{\leavevmode\hbox to \hsize{\hbox to .75\hsize
  {\hskip\keyindent\relax#1\hfil}%
  \hskip -\metaxwidth minus 1fil
  \kbd{#2}\hfil}}

% threecol - like "key" but with two key names.
% for example, one for doing the action backward, and one for forward.
\def\threecol#1#2#3{\hskip\keyindent\relax#1\hfil&\kbd{#2}\hfil\quad
  &\kbd{#3}\hfil\quad\cr}


%**end of header


\title{GNU Emacs: Cart\~ao de Refer\^encia}
\centerline{(para vers\~ao \versionemacs)}

\section{Iniciando o Emacs}

Para entrar no GNU Emacs, digite:  \kbd{emacs}

\section{Saindo do Emacs}

\key{suspende ou minimiza o Emacs}{C-z}
\key{encerra o Emacs}{C-x C-c}

\section{Arquivos}

\key{{\bf abre} um arquivo}{C-x C-f}
\key{{\bf salva} um arquivo em disco}{C-x C-s}
\key{salva {\bf todos} arquivos abertos}{C-x s}
\key{{\bf insere} outro arquivo neste buffer}{C-x i}
\key{substitui este arquivo por outro}{C-x C-v}
\key{salva o buffer em um arquivo especificado}{C-x C-w}
\key{alterna o estado de somente leitura do buffer}{C-x C-q}

\section{Ajuda (Help)}

Tecle \kbd{C-h} (ou \kbd{F1}) e siga as instru{\c{c}}{\~o}es. 

\key{remove a janela de ajuda}{C-x 1}
\key{rola a janela de ajuda}{C-M-v}

\key{apropos: mostra comandos que casam com a string}{C-h a}
\key{descreve fun{\c{c}}{\~a}o associada a teclas}{C-h k}
\key{descreve uma fun{\c{c}}{\~a}o}{C-h f}
\key{busca informa{\c{c}}{\~o}es espec{\'\i}ficas do modo}{C-h m}

\section{Recuperando-se de Erros}

\key{{\bf aborta} uma opera{\c{c}}{\~a}o}{C-g}
\metax{{\bf recupera} arquivos ap{\'o}s crash}{M-x recover-session}
\metax{desfaz uma altera{\c{c}}{\~a}o ({\bf undo})}{C-x u, C-_ {\rm or} C-/}
\metax{restaura um buffer para o arquivo}{M-x revert-buffer}
\key{redesenha a tela}{C-l}

\section{Busca Incremental}

\key{busca para frente}{C-s}
\key{busca para tr{\'a}s}{C-r}
\key{busca por express{\~a}o regular}{C-M-s}
\key{busca por express{\~a}o regular para tr{\'a}s}{C-M-r}

\key{seleciona a string de pesquisa anterior}{M-p}
\key{seleciona a string seguinte de pesquisa}{M-n}
\key{sai da busca incremental}{RET}
\key{desfaz o efeito do {\'u}ltimo caracter}{DEL}
\key{encerra a busca}{C-g}

Use \kbd{C-s} ou \kbd{C-r} novamente para repetir a busca.
\kbd{C-g} cancela apenas o que ainda n{\~a}o foi feito.

\shortcopyrightnotice

\section{Movimenta{\c{c}}{\~a}o}

\paralign to \hsize{#\tabskip=10pt plus 1 fil&#\tabskip=0pt&#\cr
\threecol{{\bf avan{\c{c}}ar}}{{\bf tr{\'a}s}}{{\bf frente}}
\threecol{um caracter}{C-b}{C-f}
\threecol{uma palavra}{M-b}{M-f}
\threecol{uma linha}{C-p}{C-n}
\threecol{para in{\'\i}cio ou fim de linha}{C-a}{C-e}
\threecol{senten{\c{c}}a}{M-a}{M-e}
\threecol{par{\'a}grafo}{M-\{}{M-\}}
\threecol{p{\'a}gina}{C-x [}{C-x ]}
\threecol{sexp}{C-M-b}{C-M-f}
\threecol{fun{\c{c}}{\~a}o}{C-M-a}{C-M-e}
\threecol{para in{\'\i}cio ou fim do buffer}{M-<}{M->}
}

\key{rolar para pr{\'o}xima tela}{C-v}
\key{rolar para tela anterior}{M-v}
\key{rolar para esquerda}{C-x <}
\key{rolar para direita}{C-x >}
\key{rolar a linha corrente para o centro da tela}{C-u C-l}

\section{Cortando e Apagando}

\paralign to \hsize{#\tabskip=10pt plus 1 fil&#\tabskip=0pt&#\cr
\threecol{{\bf entidade a cortar}}{{\bf tr{\'a}s}}{{\bf frente}}
\threecol{caracter (apaga, n{\~a}o corta)}{DEL}{C-d}
\threecol{palavra}{M-DEL}{M-d}
\threecol{linha (at{\'e} o final)}{M-0 C-k}{C-k}
\threecol{senten{\c{c}}a}{C-x DEL}{M-k}
\threecol{sexp}{M-- C-M-k}{C-M-k}
}

\key{corta {\bf regi{\~a}o}}{C-w}
\key{copia a {\bf regi{\~a}o}}{M-w}
\key{cortar at{\'e} a pr{\'o}xima ocorr{\^e}ncia de {\it char}}{M-z {\it char}}

\key{colar a {\'u}ltima coisa cortada}{C-y}
\key{substitui a {\'u}lt. colagem pela c{\'o}pia anterior}{M-y}

\section{Marcando}

\key{posiciona a marca aqui}{C-@ {\rm or} C-SPC}
\key{troca a marca pelo ponto e vice-versa}{C-x C-x}

\key{coloca a marca {\it arg\/} {\bf palavras} adiante}{M-@}
\key{marca o {\bf par{\'a}grafo}}{M-h}
\key{marca a {\bf p{\'a}gina}}{C-x C-p}
\key{marca a {\bf sexp}}{C-M-@}
\key{marca uma {\bf fun{\c{c}}{\~a}o}}{C-M-h}
\key{marca todo {\bf buffer}}{C-x h}

\section{Busca e Substitui{\c{c}}{\~a}o}

\key{Substitui interativamente uma string}{M-\%}
% query-replace-regexp is bound to C-M-% but that can't be typed on
% consoles.
\metax{usando express{\~a}o regular}{M-x query-replace-regexp}

Respostas v{\'a}lidas no modo de busca e substitui{\c{c}}{\~a}o

\key{{\bf substitui} esta, e prossegue}{SPC}
\key{substitui esta e entrada e n{\~a}o avan{\c{c}}a}{,}
\key{{\bf pula} para a pr{\'o}xima sem substituir}{DEL}
\key{substitui em todo o texto restante}{!}
\key{{\bf volta} para a palavra anterior}{^}
\key{{\bf encerra}}{RET}
\key{entra na edi{\c{c}}{\~a}o recursiva (\kbd{C-M-c} para sair)}{C-r}


\section{M{\'u}ltiplas Janelas}

Quando forem mostrados 2 comandos, o segundo tem comportamento similar
para frame.

{\setbox0=\hbox{\kbd{0}}\advance\hsize by 0\wd0
\paralign to \hsize{#\tabskip=10pt plus 1 fil&#\tabskip=0pt&#\cr
\threecol{elimina todas outras janelas}{C-x 1\ \ \ \ }{C-x 5 1}
\threecol{divide a janela, acima e abaixo}{C-x 2\ \ \ \ }{C-x 5 2}
\threecol{elimina esta janela}{C-x 0\ \ \ \ }{C-x 5 0}
}}
\key{divide a janela, lado a lado}{C-x 3}

\key{rola a outra janela}{C-M-v}

{\setbox0=\hbox{\kbd{0}}\advance\hsize by 2\wd0
\paralign to \hsize{#\tabskip=10pt plus 1 fil&#\tabskip=0pt&#\cr
\threecol{leva o cursor para outra janela}{C-x o}{C-x 5 o}

\threecol{seleciona um buffer em outra janela}{C-x 4 b}{C-x 5 b}
\threecol{mostra um buffer em outra janela}{C-x 4 C-o}{C-x 5 C-o}
\threecol{busca um arquivo em outra janela}{C-x 4 f}{C-x 5 f}
\threecol{busca arquivo (ro) em outra janela}{C-x 4 r}{C-x 5 r}
\threecol{executa Dired em outra janela}{C-x 4 d}{C-x 5 d}
\threecol{busca tag em outra janela}{C-x 4 .}{C-x 5 .}
}}

\key{aumenta a janela na vertical}{C-x ^}
\key{estreita a janela}{C-x \{}
\key{alarga a janela}{C-x \}}

\section{Formatando}

\key{identa a {\bf linha} corrente (modo)}{TAB}
\key{identa a {\bf regi{\~a}o} (modo)}{C-M-\\}
\key{identa a {\bf sexp} (modo)}{C-M-q}
\key{identa regi{\~a}o rigidamente {\it arg\/} colunas}{C-x TAB}

\key{insere uma nova linha ap{\'o}s o ponto}{C-o}
\key{move o restante da linha para baixo}{C-M-o}
\key{apaga linhas em branco em torno do ponto}{C-x C-o}
\key{junta a linha com a anterior}{M-^}
\key{apaga todos brancos em torno do ponto}{M-\\}
\key{insere um espa{\c{c}}o em branco}{M-SPC}

\key{preenche o par{\'a}grafo}{M-q}
\key{define a coluna limite de preenchimento}{C-x f}
\key{define um prefixo para cada linha}{C-x .}

\key{formata fonte}{M-o}

\section{Mai{\'u}sculas e Min{\'u}sculas}

\key{Palavra para mai{\'u}sculas}{M-u}
\key{Palavra para min{\'u}sculas}{M-l}
\key{Primeira letra mai{\'u}scula (capitalize)}{M-c}

\key{Regi{\~a}o para mai{\'u}sculas}{C-x C-u}
\key{Regi{\~a}o para min{\'u}sculas}{C-x C-l}

\section{O Minibuffer}

As teclas seguintes s{\~a}o definidas no minibuffer.

\key{complete o m{\'a}ximo possi{\'\i}vel}{TAB}
\key{complete at{\'e} uma palavra}{SPC}
\key{complete e execute}{RET}
\key{mostre as op{\c{c}}{\~o}es para completar}{?}
\key{busca a entrada anterior no minibuffer}{M-p}
\key{busca a pr{\'o}xima entrada no minibuffer ou o default}{M-n}
\key{busca regexp no hist{\'o}rico para tr{\'a}s}{M-r}
\key{busca regexp no hist{\'o}rico para frente}{M-s}
\key{encerra o comando}{C-g}

Tecle \kbd{C-x ESC ESC} para editar e repetir o {\'u}ltimo comando
utilizado.  Tecle \kbd{F10} para ativar o menu.

\newcolumn
\title{GNU Emacs: Cart\~ao de Refer\^encia}
\centerline{(para vers\~ao \versionemacs)}

\section{Buffers}

\key{seleciona outro buffer}{C-x b}
\key{lista todos buffers}{C-x C-b}
\key{mata um buffer}{C-x k}

\section{Transposi{\c{c}}{\~a}o}

\key{transp{\~o}e {\bf caracteres}}{C-t}
\key{transp{\~o}e {\bf palavras}}{M-t}
\key{transp{\~o}e {\bf linhas}}{C-x C-t}
\key{transp{\~o}e {\bf sexps}}{C-M-t}

\section{Verifica{\c{c}}{\~a}o Ortogr{\'a}fica}

\key{verifica a palavra corrente}{M-\$}
\metax{verifica todas palavras de uma regi{\~a}o}{M-x ispell-region}
\metax{verifica todo o buffer}{M-x ispell-buffer}

\section{Tags}

\key{busca uma tag (uma defini{\c{c}}{\~a}o)}{M-.}
\key{encontra a pr{\'o}xima ocorr{\^e}ncia da tag}{C-u M-.}
\metax{especifica um novo arquivo de tags}{M-x visit-tags-table}

\metax{busca por regexp em todos arquivos}{M-x tags-search}
\metax{busca e subst. em todos arquivos}{M-x tags-query-replace}
\key{continua a {\'u}ltima busca ou busca e substitui{\c{c}}{\~a}o}{M-,}

\section{Shells}

\key{executa um comando do shell}{M-!}
\key{executa um comando do shell na regi{\~a}o}{M-|}
\key{filtra uma regi{\~a}o por um comando do shell}{C-u M-|}
\key{inicia um shell na janela \kbd{*shell*}}{M-x shell}

\section{Ret{\^a}ngulos}

\key{copia o ret{\^a}ngulo para o registrador}{C-x r r}
\key{corta o ret{\^a}ngulo}{C-x r k}
\key{cola o ret{\^a}ngulo}{C-x r y}
\key{abre o ret{\^a}ngulo, move o texto para direita}{C-x r o}
\key{troca por espa{\c{c}}os o conte{\'u}do do ret{\^a}ngulo}{C-x r c}
\key{antep{\~o}e uma linha a string}{C-x r t}

\section{Abreviaturas}

\key{adiciona uma abreviatura global}{C-x a g}
\key{adiciona abreviatura ao modo local}{C-x a l}
\key{adiciona globalmente expans{\~a}o de abrev.}{C-x a i g}
\key{adiciona localmente expans{\~a}o de abrev.}{C-x a i l}
\key{explicitamente expande uma abrev}{C-x a e}

\key{completa com base em palavras anteriores}{M-/}


\section{Express{\~o}es Regulares}

\key{qualquer caracter exceto nova linha}{. {\rm(dot)}}
\key{zero ou mais repeti{\c{c}}{\~o}es}{*}
\key{uma ou mais repeti{\c{c}}{\~o}es}{+}
\key{zero ou uma repeti{\c{c}}{\~a}o}{?}
\key{protege o caracter especial {\it c\/}}{\\{\it c}}
\key{(``or'')}{\\|}
\key{agrupamento}{\\( {\rm$\ldots$} \\)}
\key{mesmo texto que {\it n\/}-{\'e}simo grupo}{\\{\it n}}
\key{quebra de palavra}{\\b}
\key{sem quebra de palavra}{\\B}

\paralign to \hsize{#\tabskip=10pt plus 1 fil&#\tabskip=0pt&#\cr
\threecol{{\bf entidade}}{{\bf casa in{\'\i}cio}}{{\bf casa fim}}
\threecol{linha}{^}{\$}
\threecol{palavra}{\\<}{\\>}
\threecol{buffer}{\\`}{\\'}

\threecol{{\bf classe de caracteres}}{{\bf casa esses}}{{\bf casa os outros}}
\threecol{conjunto expl{\'\i}cito}{[ {\rm$\ldots$} ]}{[^ {\rm$\ldots$} ]}
\threecol{caracter de sintaxe de palavra}{\\w}{\\W}
\threecol{caracter de sintaxe de {\it c}}{\\s{\it c}}{\\S{\it c}}
}

\section{Conjuntos de Carac. Internacionais}

\key{especifica uma l{\'\i}ngua principal}{C-x RET l}
\metax{mostra todos m{\'e}todos de inser{\c{c}}{\~a}o}{M-x list-input-methods}
\key{habilita/desabilita um m{\'e}todo de inser{\c{c}}{\~a}o}{C-\\}
\key{determina o sistema de codifica{\c{c}}{\~a}o}{C-x RET c}
\metax{mostra sistemas de codifica{\c{c}}{\~a}o}{M-x list-coding-systems}
\metax{escolhe a codifica{\c{c}}{\~a}o preferida}{M-x prefer-coding-system}

\section{Info}

\key{entra no leitor de Info}{C-h i}
\key{busca fun{\c{c}}{\~a}o ou arquivo no Info}{C-h S}
\beginindentedkeys

Movimenta{\c{c}}{\~a}o em um nodo:

\key{rola para frente}{SPC}
\key{rola para tr{\'a}s}{DEL}
\key{in{\'\i}cio do nodo}{. {\rm (dot)}}

Movimenta{\c{c}}{\~a}o entre nodos:

\key{{\bf pr{\'o}ximo} nodo}{n}
\key{nodo {\bf anterior}}{p}
\key{mover cima {\bf cima}}{u}
\key{seleciona item do menu pelo nome}{m}
\key{seleciona {\it n\/}-{\'e}simo item do menu}{{\it n}}
\key{segue refer{\^e}ncia cruzada  (retorna com \kbd{l})}{f}
\key{retorna {\'u}ltimo nodo visitado}{l}
\key{retorna ao diret{\'o}rio de nodos}{d}
\key{ir para o topo do arquivo Info}{t}
\key{ir para qualquer nodo por nome}{g}

Outros:

\key{executar {\bf tutorial} do Info}{h}
\key{busca pelo assunto no {\'\i}ndice}{i}
\key{busca por express{\~a}o regular}{s}
\key{{\bf sair} Info}{q}

\endindentedkeys

\section{Registrador}

\key{salva regi{\~a}o em um registrador}{C-x r s}
\key{insere o conte{\'u}do do registrador no buffer}{C-x r i}

\key{salva valor do ponto no registrador}{C-x r SPC}
\key{salta para o ponto salvo no registrador}{C-x r j}

\section{Macros de Teclado}

\key{{\bf inicia} a defini{\c{c}}{\~a}o de uma macro}{C-x (}
\key{{\bf encerra} a defini{\c{c}}{\~a}o de uma macro}{C-x )}
\key{{\bf executa} a {\'u}ltima macro definida}{C-x e}
\key{adiciona a {\'u}ltima macro definida}{C-u C-x (}
\metax{nomeia a {\'u}ltima macro definida}{M-x name-last-kbd-macro}
\metax{insere uma defini{\c{c}}{\~a}o em Lisp}{M-x insert-kbd-macro}

\section{Lidando com Emacs Lisp}

\key{avalia {\bf sexp} antes do ponto}{C-x C-e}
\key{avalia a {\bf defun} corrente}{C-M-x}
\metax{avalia a {\bf regi{\~a}o}}{M-x eval-region}
\key{l{\^e} e avalia o minibuffer}{M-:}
\metax{carrega do diret{\'o}rio padr{\~a}o do sistema}{M-x load-library}

\section{Personaliza{\c{c}}{\~a}o Simples}

\metax{personaliza vari{\'a}veis e fontes}{M-x customize}

% The intended audience here is the person who wants to make simple
% customizations and knows Lisp syntax.

Fazendo teclas de atalho globais em Emacs Lisp (exemplos):

\beginexample%
(global-set-key "\\C-cg" 'goto-line)
(global-set-key "\\M-\#" 'query-replace-regexp)
\endexample

\section{Escrevendo Comandos}

\beginexample%
(defun \<command-name> (\<args>)
  "\<documentation>" (interactive "\<template>")
  \<body>)
\endexample

Um exemplo:

\beginexample%
(defun this-line-to-top-of-window (line)
  "Reposition line point is on to top of window.
With ARG, put point on line ARG."
  (interactive "P")
  (recenter (if (null line)
                0
              (prefix-numeric-value line))))
\endexample

A especifica{\c{c}}{\~a}o \kbd{interactive} explica como ler
interativamente argumentos.  Tecle \kbd{C-h f interactive} para mais
detalhes.

\copyrightnotice

\bye

% Local variables:
% compile-command: "pdftex pt-br-refcard"
% coding: iso-latin-1
% End:

% arch-tag: 56bf248c-a1f3-443d-93f7-02d7aea67d94
