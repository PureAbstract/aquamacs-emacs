% ViperCard -- The Reference Card for Viper under GNU Emacs 21 and XEmacs 20
%**start of header
\newcount\columnsperpage

% This file can be printed with 1 or 2 columns per page (see below).
% Specify how many you want here.  Nothing else needs to be changed.

\columnsperpage=2

% Copyright (C) 1995, 1996, 1997, 2002, 2003, 2004,
%   2005 Free Software Foundation, Inc.

% This file is part of GNU Emacs.

% This file is distributed in the hope that it will be useful,
% but WITHOUT ANY WARRANTY.  No author or distributor
% accepts responsibility to anyone for the consequences of using it
% or for whether it serves any particular purpose or describes
% any piece of software unless they say so in writing.  Refer to the
% GNU Emacs General Public License for full details.
%
% Permission is granted to copy, modify and redistribute this source
% file provided the copyright notice and permission notices are
% preserved on all copies.
%
% Permission is granted to process this file through TeX and print the
% results, provided the printed document carries copyright and
% permission notices identical to the ones below.

% This file is intended to be processed by plain TeX (TeX82).
%
% The final reference card has six columns, three on each side.
% This file can be used to produce it in any of three ways:
% 1 column per page
%    produces six separate pages, each of which needs to be reduced to 80%.
%    This gives the best resolution.
% 2 columns per page
%    produces three already-reduced pages.
%    You will still need to cut and paste.
% 3 columns per page
%    produces two pages which must be printed sideways to make a
%    ready-to-use 8.5 x 11 inch reference card.
%    For this you need a dvi device driver that can print sideways.
% Which mode to use is controlled by setting \columnsperpage above.
%
% Author of Viper:
%  Michael Kifer
%  email: kifer@cs.sunysb.edu
%
% Author of VIP 4.3:
%  Aamod Sane
%  email: sane@cs.uiuc.edu
%
% Author of VIP 3.5:
%  Masahiko Sato
%  email: ms@sail.stanford.edu
%
% The original TeX code for formatting the reference card was written by:
%  Stephen Gildea
%  UUCP: mit-erl!gildea
%  email: gildea@stop.mail-abuse.org


\def\versionnumber{3.0}
\def\year{2005}
\def\version{August \year\ v\versionnumber}

\def\shortcopyrightnotice{\vskip 1ex plus 2 fill
  \centerline{\small \copyright\ \year\ Free Software Foundation, Inc.
  Permissions on back.  v\versionnumber}}

\def\copyrightnotice{
%\vskip 1ex plus 2 fill\begingroup\small
\vskip 1ex \begingroup\small
\centerline{Copyright \copyright\ \year\ Free Software Foundation, Inc.}
\centerline{by Michael Kifer, Viper \version}
\centerline{by Aamod Sane, VIP version 4.3}
\centerline{by Masahiko Sato, VIP version 3.5}

Permission is granted to make and distribute copies of
this card provided the copyright notice and this permission notice
are preserved on all copies.

For copies of the GNU Emacs manual, write to the Free Software
Foundation, Inc., 51 Franklin Street, Fifth Floor, Boston, MA 02110-1301, USA.

\endgroup}

% make \bye not \outer so that the \def\bye in the \else clause below
% can be scanned without complaint.
\def\bye{\par\vfill\supereject\end}

\newdimen\intercolumnskip
\newbox\columna
\newbox\columnb

\def\ncolumns{\the\columnsperpage}

\message{[\ncolumns\space
  column\if 1\ncolumns\else s\fi\space per page]}

\def\scaledmag#1{ scaled \magstep #1}

% This multi-way format was designed by Stephen Gildea
% October 1986.
% Slightly modified by Masahiko Sato, September 1987.
\if 1\ncolumns
  \hsize 4in
  \vsize 10in
  %\voffset -.7in
  \voffset -.57in
  \font\titlefont=\fontname\tenbf \scaledmag3
  \font\headingfont=\fontname\tenbf \scaledmag2
  \font\miniheadingfont=\fontname\tenbf \scaledmag1 % masahiko
  \font\smallfont=\fontname\sevenrm
  \font\smallsy=\fontname\sevensy

  \footline{\hss\folio}
  \def\makefootline{\baselineskip10pt\hsize6.5in\line{\the\footline}}
\else
  %\hsize 3.2in
  %\vsize 7.95in
  \hsize 3.41in % masahiko
  \vsize 8in % masahiko
  \hoffset -.75in
  \voffset -.745in
  \font\titlefont=cmbx10 \scaledmag2
  \font\headingfont=cmbx10 \scaledmag1
  \font\miniheadingfont=cmbx10 % masahiko
  \font\smallfont=cmr6
  \font\smallsy=cmsy6
  \font\eightrm=cmr8
  \font\eightbf=cmbx8
  \font\eightit=cmti8
  \font\eightsl=cmsl8
  \font\eighttt=cmtt8
  \font\eightsy=cmsy8
  \textfont0=\eightrm
  \textfont2=\eightsy
  \def\rm{\eightrm}
  \def\bf{\eightbf}
  \def\it{\eightit}
  \def\sl{\eightsl} % masahiko
  \def\tt{\eighttt}
  \normalbaselineskip=.8\normalbaselineskip
  \normallineskip=.8\normallineskip
  \normallineskiplimit=.8\normallineskiplimit
  \normalbaselines\rm           %make definitions take effect

  \if 2\ncolumns
    \let\maxcolumn=b
    \footline{\hss\rm\folio\hss}
    \def\makefootline{\vskip 2in \hsize=6.86in\line{\the\footline}}
  \else \if 3\ncolumns
    \let\maxcolumn=c
    \nopagenumbers
  \else
    \errhelp{You must set \columnsperpage equal to 1, 2, or 3.}
    \errmessage{Illegal number of columns per page}
  \fi\fi

  %\intercolumnskip=.46in
  \intercolumnskip=.19in % masahiko .19x4 + 3.41x3 = 10.99
  \def\abc{a}
  \output={%
      % This next line is useful when designing the layout.
      %\immediate\write16{Column \folio\abc\space starts with \firstmark}
      \if \maxcolumn\abc \multicolumnformat \global\def\abc{a}
      \else\if a\abc
        \global\setbox\columna\columnbox \global\def\abc{b}
        %% in case we never use \columnb (two-column mode)
        \global\setbox\columnb\hbox to -\intercolumnskip{}
      \else
        \global\setbox\columnb\columnbox \global\def\abc{c}\fi\fi}
  \def\multicolumnformat{\shipout\vbox{\makeheadline
      \hbox{\box\columna\hskip\intercolumnskip
        \box\columnb\hskip\intercolumnskip\columnbox}
      \makefootline}\advancepageno}
  \def\columnbox{\leftline{\pagebody}}

  \def\bye{\par\vfill\supereject
    \if a\abc \else\null\vfill\eject\fi
    \if a\abc \else\null\vfill\eject\fi
    \end}
\fi

% we won't be using math mode much, so redefine some of the characters
% we might want to talk about
\catcode`\^=12
\catcode`\_=12

\chardef\\=`\\
\chardef\{=`\{
\chardef\}=`\}

\hyphenation{mini-buf-fer}

\parindent 0pt
\parskip 1ex plus .5ex minus .5ex

\def\small{\smallfont\textfont2=\smallsy\baselineskip=.8\baselineskip}

\outer\def\newcolumn{\vfill\eject}

\outer\def\title#1{{\titlefont\centerline{#1}}\vskip 1ex plus .5ex}

\outer\def\section#1{\par\filbreak
  \vskip 3ex plus 2ex minus 2ex {\headingfont #1}\mark{#1}%
  \vskip 2ex plus 1ex minus 1.5ex}

% masahiko
\outer\def\subsection#1{\par\filbreak
  \vskip 2ex plus 2ex minus 2ex {\miniheadingfont #1}\mark{#1}%
  \vskip 1ex plus 1ex minus 1.5ex}

\newdimen\keyindent

\def\beginindentedkeys{\keyindent=1em}
\def\endindentedkeys{\keyindent=0em}
\endindentedkeys

\def\paralign{\vskip\parskip\halign}

\def\<#1>{$\langle${\rm #1}$\rangle$}

\def\kbd#1{{\tt#1}\null}        %\null so not an abbrev even if period follows

\def\beginexample{\par\leavevmode\begingroup
  \obeylines\obeyspaces\parskip0pt\tt}
{\obeyspaces\global\let =\ }
\def\endexample{\endgroup}

\def\key#1#2{\leavevmode\hbox to \hsize{\vtop
  {\hsize=.75\hsize\rightskip=1em
  \hskip\keyindent\relax#1}\kbd{#2}\hfil}}

\newbox\metaxbox
\setbox\metaxbox\hbox{\kbd{M-x }}
\newdimen\metaxwidth
\metaxwidth=\wd\metaxbox

\def\metax#1#2{\leavevmode\hbox to \hsize{\hbox to .75\hsize
  {\hskip\keyindent\relax#1\hfil}%
  \hskip -\metaxwidth minus 1fil
  \kbd{#2}\hfil}}

\def\fivecol#1#2#3#4#5{\hskip\keyindent\relax#1\hfil&\kbd{#2}\quad
  &\kbd{#3}\quad&\kbd{#4}\quad&\kbd{#5}\cr}

\def\fourcol#1#2#3#4{\hskip\keyindent\relax#1\hfil&\kbd{#2}\quad
  &\kbd{#3}\quad&\kbd{#4}\quad\cr}

\def\threecol#1#2#3{\hskip\keyindent\relax#1\hfil&\kbd{#2}\quad
  &\kbd{#3}\quad\cr}

\def\twocol#1#2{\hskip\keyindent\relax\kbd{#1}\hfil&\kbd{#2}\quad\cr}

\def\twocolkey#1#2#3#4{\hskip\keyindent\relax#1\hfil&\kbd{#2}\quad&\relax#3\hfil&\kbd{#4}\quad\cr}

%**end of header

\beginindentedkeys

\title{ViperCard: Viper Reference Pal}

\centerline{(Version 3.0 (Polyglot) for Emacs 21 and XEmacs 20)}

%\copyrightnotice

\section{Loading Viper}

Just type \kbd{M-x viper-mode} followed by \kbd{RET}

OR put

(setq viper-mode t)
(require 'viper)

in .emacs

\section{Viper States}

Viper has four states: {\it emacs state}, {\it vi state}, {\it insert state},
{\it replace state}.
Mode line tells you which state you are in.
In emacs state you can do all the normal GNU Emacs editing.
This card explains only vi state and insert state (replace state is similar
to insert state).
{\bf GNU Emacs Reference Card} explains emacs state.
You can switch states as follows.

\key{from emacs state to vi state}{C-z}
\key{from vi state to emacs state}{C-z}
\key{from vi state to emacs state for 1 command}{$\backslash$}
\metax{from vi state to insert state}{i, I, a, A, o, O}
\metax{from vi state to replace state}{c, C, R}
\key{from insert or replace state to vi state}{ESC}
\key{from insert state to vi state for 1 command}{C-z}


\section{Insert Mode}
You can do editing in insert state.

\metax{go back to vi state}{ESC}
\metax{delete previous character}{C-h, DEL}
\key{delete previous word}{C-w}
\key{delete line word}{C-u}
\key{indent shiftwidth forward}{C-t}
\key{indent shiftwidth backward}{C-d}
\key{delete line word}{C-u}
\key{quote following character}{C-v}
\key{emulate Meta key in emacs state}{C-$\backslash$}
\key{escape to Vi state for one command}{C-z}

\vskip 2mm

{\bf The rest of this card explains commands in {\bf vi state}.}

\section{Getting Information on Viper}

Execute info command by typing \kbd{M-x info} and select menu item
\kbd{viper}.  Also:

\key{describe function attached to the key {\it x}}{$\backslash$ C-h k {\it x}}

\section{Leaving Emacs}

\metax{suspend Emacs}{:st {\rm or} :su}
\metax{exit Emacs permanently}{C-xC-c}
\metax{exit current file}{:wq {\rm or} :q}

\shortcopyrightnotice

\section{Error Recovery}

\metax{abort command}{C-c (user level = 1)}
\metax{abort command}{C-g (user level > 1)}
\key{redraw messed up screen}{C-l}
\metax{{\bf recover} after system crash}{:rec file}
\metax{restore a buffer }{:e!\ {\rm or} M-x revert-buffer}


\section{Counts}

Most commands in vi state accept a {\it count} which can be supplied as a
prefix to the commands.  In most cases, if a count is given, the
command is executed that many times.  E.g., \kbd{5 d d} deletes 5
lines.

\section{Registers}

There are 26 registers (\kbd{a} to \kbd{z}) that can store texts
and marks.
You can append a text at the end of a register (say \kbd{x}) by
specifying the register name in capital letter (say \kbd{X}).
There are also 9 read only registers (\kbd{1} to \kbd{9}) that store
up to 9 previous changes.
We will use {\it x\/} to denote a register.
\section{Entering Insert Mode}

\key{{\bf insert} at point}{i}
\key{{\bf append} after cursor}{a}
\key{{\bf insert} before first non-white}{I}
\key{{\bf append} at end of line}{A}
\key{{\bf open} line below}{o}
\key{{\bf open} line above}{O}

\section{Buffers and Windows}

\key{move cursor to {\bf next} window}{C-x o}
\key{delete current window}{C-x 0}
\key{delete other windows}{C-x 1}
\key{split current window into two windows}{C-x 2}
\key{{\bf switch} to a buffer in the current window}{C-x {\sl buffer}}
\metax{{\bf switch} to a buffer in another window}{:n, :b, {\rm or} C-x 4 {\sl buf}}
\key{{\bf kill} a buffer}{:q! {\rm or} C-x k}
\key{list existing {\bf buffers}}{:args {\rm or} C-x b}

\section{Files}

\metax{{\bf visit} file in the current window}{v {\sl file} {\rm or} :e {\sl file}}
\key{{\bf visit} file in another window}{V {\sl file}}
\key{{\bf visit} file in another frame}{C-v {\sl file}}
\key{{\bf save} buffer to the associated file}{:w {\rm or} C-xC-s}
\metax{{\bf write} buffer to a specified file}{:w {\sl file} {\rm or} C-xC-w}
\metax{{\bf insert} a specified file at point}{:r {\sl file} {\rm or} C-xi}
\key{{\bf get} information on the current {\bf file}}{C-c g {\rm or} :f}
\key{run the {\bf directory} editor}{:e RET {\rm or} C-xd}

%\shortcopyrightnotice

\section{Viewing the Buffer}

\key{scroll to next screen}{C-f}
\key{scroll to previous screen}{C-b}
\key{scroll {\bf down} half screen}{C-d}
\key{scroll {\bf up} half screen}{C-u}
\key{scroll down one line}{C-e}
\key{scroll up one line}{C-y}

\key{put current line on the {\bf home} line}{z H {\rm or} z RET}
\key{put current line on the {\bf middle} line}{z M {\rm or} z .}
\key{put current line on the {\bf last} line}{z L {\rm or} z -}

\section{Marking and Returning}

\key{{\bf mark} point in register {\it x}}{m {\it x}}
\key{set mark at buffer beginning}{m <}
\key{set mark at buffer end}{m >}
\key{set mark at point}{m .}
\key{jump to mark}{m ,}
\key{exchange point and mark}{` `}
\key{... and skip to first non-white on line}{' '}
\key{go to mark {\it x}}{` {\it x}}
\key{... and skip to first non-white on line}{' {\it x}}
\key{view contents of marker {\it x}}{[ {\it x}}
\key{view contents of register {\it x}}{] {\it x}}

\section{Macros}

Emacs style macros:

\key{start remembering keyboard macro}{C-x (}
\key{finish remembering keyboard macro}{C-x )}
\key{call last keyboard macro}{*}

\key{start remembering keyboard macro}{@ \#}
\key{finish macro and put into register {\it x}}{@ {\it x}}
\key{execute macro stored in register {\it x}}{@ {\it x}}
\key{repeat last @{\it x} command}{@ @}

\key{Pull last macro into register {\it x}}{@ ! {\it x}}

Vi-style macros (keys to be hit in quick succession):

\key{define Vi-style macro for Vi state}{:map}
\key{define Vi-style macro for Insert state}{:map!}

\key{toggle case-sensitive search}{//}
\key{toggle regular expression search}{///}
\key{toggle `\%' to ignore parentheses inside comments}{\%\%\%}


\section{Motion Commands}

\key{go backward one character}{h {\rm or} C-h}
\key{go forward one character}{l}
\metax{next line keeping the column}{j {\rm or} LF {\rm or} C-n}
\key{previous line keeping the column}{k}
\metax{next line at first non-white}{+ {\rm or} RET {\rm or} C-p}
\key{previous line at first non-white}{-}

\key{beginning of line}{0}
\key{first non-white on line}{^}
\key{end of line}{\$}
\key{go to {\it n}-th column on line}{{\it n} |}

\key{go to {\it n}-th line}{{\it n} G}
\key{go to last line}{G}
\key{find matching parenthesis for \kbd{()}, \kbd{\{\}} and \kbd{[]}}{\%}

\key{go to {\bf home} window line}{H}
\key{go to {\bf middle} window line}{M}
\key{go to {\bf last} window line}{L}

\subsection{Words, Sentences, Paragraphs, Headings}

\key{forward {\bf word}}{w {\rm or} W}
\key{{\bf backward} word}{b {\rm or} B}
\key{{\bf end} of word}{e {\rm or} E}

In the case of capital letter commands, a word is delimited by a
non-white character.

\key{forward sentence}{)}
\key{backward sentence}{(}

\key{forward paragraph}{\}}
\key{backward paragraph}{\{}

\key{forward heading}{]]}
\key{backward heading}{[[}
\key{end of heading}{[]}

\subsection{Find Characters on the Line}

\key{{\bf find} {\it c} forward on line}{f {\it c}}
\key{{\bf find} {\it c} backward on line}{F {\it c}}
\key{up {\bf to} {\it c} forward on line}{t {\it c}}
\key{up {\bf to} {\it c} backward on line}{T {\it c}}
\key{repeat previous \kbd{f}, \kbd{F}, \kbd{t} or \kbd{T}}{;}
\key{... in the opposite direction}{,}

%\newcolumn
%\title{Viper Quick Reference Card}

\section{Searching and Replacing}

\key{search forward for {\sl pat}}{/ {\sl pat}}
\key{search backward with previous {\sl pat}}{?\ RET}
\key{search forward with previous {\sl pat}}{/ RET}
\key{search backward for {\sl pat}}{?\ {\sl pat}}
\key{repeat previous search}{n}
\key{... in the opposite direction}{N}

\key{{\bf query} replace}{Q}
\key{{\bf replace} a character by another character {\it c}}{r {\it c}}
\key{{\bf overwrite} {\it n} lines}{{\it n} R}

\metax{{\bf buffer} search (if enabled)}{g {\it move command}}

\section{Modifying Commands}

Most commands that operate on text regions accept the motion commands,
to describe regions. They also accept the Emacs region specifications
{\bf r} and {\bf R}. {\bf r} describes the region between {\it point}
and {\it mark}, and {\bf R} describes whole lines in that region.
Motion commands are classified into {\it point commands} and
{\it line commands}.  In the case of line commands, whole lines will
be affected by the command.

The point commands are as follows:

\hskip 5ex
\kbd{h l 0 ^ \$ w W b B e E ( ) / ?\ ` f F t T \% ; ,}

The line commands are as follows:

\hskip 5ex
\kbd{j k + - H M L \{ \} G '}

These region specifiers will be referred to as {\it m} below.

\subsection{Delete/Yank/Change Commands}

\paralign to \hsize{#\tabskip=10pt plus 1 fil&#\tabskip=0pt&#\tabskip=0pt&#\cr
\fourcol{}{{\bf delete}}{{\bf yank}}{{\bf change}}
\fourcol{region determined by {\it m}}{d {\it m}}{y {\it m}}{c {\it m}}
\fourcol{... into register {\it x}}{" {\it x\/} d {\it m}}{" {\it x\/} y {\it m}}{" {\it x\/} c {\it m}}
\fourcol{a line}{d d}{Y {\rm or} y y}{c c}
\fourcol{current {\bf region}}{d r}{y r}{c r}
\fourcol{expanded {\bf region}}{d R}{y R}{c R}
\fourcol{to end of line}{D}{y \$}{c \$}
\fourcol{a character after point}{x}{y l}{c l}
\fourcol{a character before point}{DEL}{y h}{c h}
}

\vskip 2ex
\key{Overwrite {\it n} lines}{{\it n} R}

\subsection{Put Back Commands}

Deleted/yanked/changed text can be put back by the following commands.

\key{{\bf Put} back at point/above line}{P}
\key{... from register {\it x}}{" {\it x\/} P}
\key{{\bf put} back after point/below line}{p}
\key{... from register {\it x}}{" {\it x\/} p}

\subsection{Repeating and Undoing Modifications}

\key{{\bf undo} last change}{u {\rm or} :und}
\key{repeat last change}{.\ {\rm (dot)}}

Undo is undoable by \kbd{u} and repeatable by \kbd{.}.
For example, \kbd{u...} will undo 4 previous changes.
A \kbd{.} after \kbd{5dd} is equivalent to \kbd{5dd},
while \kbd{3.} after \kbd{5dd} is equivalent to \kbd{3dd}.

\section{Miscellaneous Commands}

\endindentedkeys

\paralign to \hsize{#\tabskip=5pt plus 1 fil&#\tabskip=0pt&#\tabskip=0pt&#\tabskip=0pt&#\cr
\fivecol{}{{\bf shift left}}{{\bf shift right}}{{\bf filter shell command}}{{\bf indent}}
\fivecol{region}{< {\it m}}{> {\it m}}{!\ {\it m\/} {\sl shell-com}}{= {\it m}}
\fivecol{line}{< <}{> >}{!\ !\ {\sl shell-com}}{= =}
}

\key{{\bf join} lines}{J}
\key{toggle case (takes count)}{\~{}}

\key{view register {\it x}}{] {\it x}}
\key{view marker {\it x}}{] {\it x}}

\key{lowercase region}{\# c {\it m}}
\key{uppercase region}{\# C {\it m}}
\key{execute last keyboard macro on each line in the region}{\# g {\it m}}

\key{insert specified string for each line in the region}{\# q {\it m}}
\key{check spelling of the words in the region}{\# s {\it m}}

\key{repeat previous ex substitution}{\&}
\key{change to previous file}{C-^}

\key{Viper Meta key}{_}

\section{Customization}

By default, search is case sensitive.
You can change this by including the following line in your \kbd{\~{}/.vip} file.

\hskip 5ex
\kbd{(setq viper-case-fold-search t)}

The following is a subset of the variety of
options available for customizing Viper.
See the Viper manual for details on these and other options.

\beginindentedkeys

\paralign to \hsize{#\tabskip=10pt plus 1 fil&#\tabskip=0pt&#\cr
\twocol{{\bf variable}}{{\bf default value}}
\twocol{viper-search-wrap-around}{t}
\twocol{viper-case-fold-search}{nil}
\twocol{viper-re-search}{t}
\twocol{viper-re-replace}{t}
\twocol{viper-re-query-replace}{t}
\twocol{viper-auto-indent}{nil}
\twocol{viper-shift-width}{8}
\twocol{viper-tags-file-name}{"TAGS"}
\twocol{viper-no-multiple-ESC}{t}
\twocol{viper-ex-style-motion}{t}
\twocol{viper-always}{t}
\twocol{viper-custom-file-name}{"\~{}/.vip"}
\twocol{ex-find-file-shell}{"csh"}
\twocol{ex-cycle-other-window}{t}
\twocol{ex-cycle-through-non-buffers}{t}
\twocol{blink-matching-paren}{t}
\twocol{buffer-read-only}{{\it buffer dependent}}
}

To bind keys in Vi command state, put lines like these in your
\kbd{\~{}/.vip} file:

\beginexample
(define-key viper-vi-global-user-map "\\C-v" 'scroll-down)
(define-key viper-vi-global-user-map "\\C-cm" 'smail)
\endexample


\newcolumn

\title{Ex Commands in Viper}

In vi state, an Ex command is entered by typing:

\hskip 5ex
\kbd{:\ {\sl ex-command} RET}

\section{Ex Addresses}

\paralign to \hsize{#\tabskip=5pt plus 1 fil&#\tabskip=2pt&#\tabskip=5pt plus 1 fil&#\cr
\twocolkey{current line}{.}{next line with {\sl pat}}{/ {\sl pat} /}
\twocolkey{line {\it n}}{{\it n}}{previous line with {\sl pat}}{?\ {\sl pat} ?}
\twocolkey{last line}{\$}{{\it n\/} line before {\it a}}{{\it a} - {\it n}}
\twocolkey{next line}{+}{{\it a\/} through {\it b}}{{\it a\/} , {\it b}}
\twocolkey{previous line}{-}{line marked with {\it x}}{' {\it x}}
\twocolkey{entire buffer}{\%}{previous context}{' '}
}

Addresses can be specified in front of a command.
For example,

\hskip 5ex
\kbd{:.,.+10m\$}

moves 11 lines below current line to the end of buffer.

\section{Ex Commands}

Avoid Ex text manipulation commands except substitute.
There are better VI equivalents
for all of them. Also note that all Ex commands expand \% to
current file name. To include a \% in the command, escape it with a $\backslash$.
Similarly, \# is replaced by previous file. For Viper, this is the
first  file in the {\sl :args} listing for that buffer. This defaults
to the previous file in the VI sense if you have one window.
Ex commands can be made to have history. See the manual for details.

\subsection{Ex Text Commands}

\endindentedkeys

\key{mark lines matching {\sl pat} and execute {\sl cmds} on these lines}{:g /{\sl pat}/ {\sl cmds}}

\key{mark lines {\it not\/} matching {\sl pat} and execute {\sl cmds} on these lines}{:v /{\sl pat}/ {\sl cmds}}


\key{{\bf move} specified lines after {\sl addr}}{:m {\sl addr}}
\key{{\bf copy} specified lines after {\sl addr}}{:co\rm\ (or \kbd{:t})\ \sl addr}
\key{{\bf delete} specified lines [into register {\it x\/}]}{:d {\rm [{\it x\/}]}}
\key{{\bf yank} specified lines [into register {\it x\/}]}{:y {\rm [{\it x\/}]}}
\key{{\bf put} back text [from register {\it x\/}]}{:pu {\rm [{\it x\/}]}}

\key{{\bf substitute} {\sl repl} for first string on line matching {\sl pat}}{:s /{\sl pat}/{\sl repl}/}

\key{repeat last substitution}{:\&}
\key{repeat previous substitute with previous search pattern as {\sl pat}}{:\~{}}

\subsection{Ex File and Shell Commands}

\key{{\bf edit} file}{:e {\sl file}}
\key{reedit messed up current file}{:e!}
\key{edit previous file}{:e\#}
\key{{\bf read} in a file}{:r {\sl file}}
\key{{\bf read} in the output of a shell command}{:r {\sl !command}}
\key{write out specified lines into {\sl file}}{:w {\sl file}}
\key{save all modified buffers, ask confirmation}{:W {\sl file}}
\key{save all modified buffers, no confirmation}{:WW {\sl file}}
\key{write out specified lines at the end of {\sl file}}{:w>> {\sl file}}
\key{{\bf write} to the input of a shell command}{:w {\sl !command}}
\key{write out and then quit}{:wq {\sl file}}

\key{run a sub{\bf shell} in a window}{:sh}
\key{execute shell command {\sl command}}{:!\ {\sl command}}
\key{execute previous shell command with {\it args} appended}{:!! {\sl args}}

\subsection{Ex Miscellaneous Commands}

\key{define a macro {\it x} that expands to {\sl cmd}}{:map {\it x} {\sl cmd}}
\key{remove macro expansion associated with {\it x}}{:unma {\it x}}
\key{define a macro {\it x} that expands to {\sl cmd} in insert state}{:map!\ {\it x} {\sl cmd}}
\key{remove macro expansion associated with {\it x} in insert state}{:unma!\ {\it x}}

\key{print line number}{:.=}
\key{print last line number}{:=}
\key{print {\bf version} number of Viper}{:ve}

\key{shift specified lines to the right}{:>}
\key{shift specified lines to the left}{:<}

\key{{\bf join} lines}{:j}
\key{mark specified line to register {\it x}}{:k {\it x}}
\key{{\bf set} a variable's value}{:se}
\key{find first definition of {\bf tag} {\sl tag}}{:ta {\sl tag}}

\key{Current directory}{:pwd}


\copyrightnotice

\bye

% Local variables:
% compile-command: "tex viperCard"
% End:

% arch-tag: e287b45c-4c5e-4bf9-ae28-ead4cd9f68e3
